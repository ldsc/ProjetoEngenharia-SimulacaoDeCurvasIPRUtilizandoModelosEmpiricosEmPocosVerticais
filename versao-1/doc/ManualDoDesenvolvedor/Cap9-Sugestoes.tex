
\chapter{Sugest�es para Trabalhos Futuros}

\lhead{\thechapter - Sugest�es para Trabalhos Futuros} 

Com a finalidade de melhorar o projeto, sugere-se que os t�picos abaixo
sejam incorporados de modo que o c�digo e formula��o te�rica sejam
ampliados a outros casos.
\begin{itemize}
\item Espeficica��o:
\begin{itemize}
\item Apresenta-se a seguir os requisitos funcionais.
\end{itemize}
\begin{tabular*}{14cm}{@{\extracolsep{\fill}}|c|p{11.5cm}|}
\hline 
\textbf{RF-01} & O usu�rio deve ter liberdade para escolher quais par�metros de entrada
utilizar, utilizando teclado ou um arquivo .txt.\tabularnewline
\hline 
\end{tabular*}
\begin{itemize}
\item .
\end{itemize}
\item Formula��o Te�rica:
\begin{itemize}
\item Adicionar a IPR que abrange reservat�rios estratificados, ou seja,
reservat�rios com mais de uma camada e possibilitar o c�lculo utilizando
os modelos emp�ricos.
\item Adicionar a IPR Futura, m�todo utilizado para prever produtividade. 
\item Implementar outros regimes de escoamento, como o permanente e transiente.
\item Considerar os c�lculos para po�os horizontais ou direcionais.
\end{itemize}
\item Formula��o do C�digo:
\begin{itemize}
\item Permitir que o usu�rio entre com os dados na forma de arquivo de disco.
\item Permitir que o usu�rio salve os resultados de press�o e vaz�o calculados
pelo m�todo escolhido.
\end{itemize}
\end{itemize}

